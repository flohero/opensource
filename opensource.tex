\documentclass{beamer}
\usepackage[utf8]{inputenc}
\usepackage{hyperref}

\usetheme{Madrid}
\setbeamertemplate{headline}{}
\setbeamertemplate{items}[circle]

\title{Open Source}
\subtitle{Principles of Open Source}
\author{Florian Weingartshofer}
\institute{FH Hagenberg}
\date{\today}
\subject{English}

\begin{document}
\titlepage

\begin{frame}
  \frametitle{Outline}
  \tableofcontents
\end{frame}

\section{What is Open Source?}
\section{Where is it used?}

\begin{frame}
  \frametitle{What's Open Source?}
  \begin{columns}
    \begin{column}{0.5\textwidth}
%      A way of developing Software
      \begin{center}
        \Large
        Transparency

        Collaboration

        Release early and often

        Meritocracy

        Community
      \end{center}
    \end{column}
    \begin{column}{0.5\textwidth}  %%<--- here
      \begin{center}
        \begin{figure}
          \includegraphics[scale=0.03]{./img/open.jpg}
          \caption{unsplash.com/@stairhopper}
        \end{figure}
      \end{center}
    \end{column}
  \end{columns}
\end{frame}

\begin{frame}
  \frametitle{Who is contributing?}
  \begin{enumerate}
  \item Microsoft
  \item Google
  \item Red Hat
  \item IBM
  \end{enumerate}
\end{frame}


\begin{frame}
  \frametitle{Biggest Projects}
  Linux

  Fedora

  Android

  Java
\end{frame}


\begin{frame}
  \frametitle{Questions?}
  \begin{center}
    The Presentation is obviously Open Source at
    \underline{\href{https://github.com/flohero/opensource-presentation}{github.com/flohero/opensource-presentation}}
    \end{center}
\end{frame}

\begin{frame}
  \frametitle{Bibliography}
  \begin{center}
    \Large
    \href{https://opensource.com/open-source-way}{opensource.com}
    
    \href{https://twitter.com/filmaj}{twitter.com/filmaj}
    
  \end{center}
\end{frame}
\end{document}
